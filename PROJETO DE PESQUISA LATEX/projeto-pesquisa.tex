%%%%%%%%%%%%%%%%%%
%%%   Template para utilização   %%%
%%%%%%%%%%%%%%%%%%

%Para melhor utilização da ferramenta, utilize o pdfLaTeX+MakeIndex+BibTeX%

\documentclass[12pt,openright,oneside,a4paper,english,french,spanish,brazil]{unifil}

\titulo{Título} % Título da monografia
\autor{Seu Nome} % Nome do autor
\instituicao{Centro Universitário Filadélfia}
\local{Londrina}
\data{2019} % Ano de publicação
\preambulo{Nome do Curso} % Nome do curso
\orientador{Nome do Orientador} % Nome do orientador

\begin{document}

\frenchspacing

%%%%%%%%%%%%%%%%%%%%%%%%%%%%
%% Elementos pré-textuais %%
%%%%%%%%%%%%%%%%%%%%%%%%%%%%

\pretextual

\imprimircapa
\makeatletter
\renewcommand{\folhaderostocontent}{
\begin{center}
{\ABNTEXchapterfont\bfseries\MakeTextUppercase{\imprimirautor}}
\vspace*{5cm}
\begin{center}
\ABNTEXchapterfont\bfseries\normalsize\MakeTextUppercase{\imprimirtitulo}
\end{center}
\vspace*{1cm}
\abntex@ifnotempty{\imprimirpreambulo}{%
\hspace{.45\textwidth}
\begin{minipage}{.5\textwidth}
\SingleSpacing
 Trabalho de Dissertação apresentado ao \imprimirinstituicao~como parte dos requisitos para obtenção de graduação em \imprimirpreambulo.
{\textnormal{\imprimirorientadorRotulo~\imprimirorientador.}}
\end{minipage}%
}%
\vspace*{\fill}
{\bfseries\large\imprimirlocal}
\par
{\bfseries\large\imprimirdata}
\vspace*{1cm}
\end{center}
}
\makeatother

\imprimirfolhaderosto

\clearpage{\pagestyle{empty}\cleardoublepage}
%%Altere as informações do resumo%%
\noindent{SOBRENOME; NOME, D. \textbf{\imprimirtitulo}. Trabalho de Conclusão de Curso (Graduação) - \imprimirinstituicao. \imprimirlocal, \imprimirdata.}
\par
\begin{resumo}
	%%Escreva seu resumo na língua vernácula aqui%%
\lipsum[5]
\vspace{\onelineskip} 
	%%Adicione as palavras chaves após os dois pontos '':''%%
\noindent
\textbf{Palavras-chaves}: 3 ou mais.

\end{resumo}


%\par
%\vspace{11cm}

\tableofcontents*

  \setlength\absleftindent{0cm}
  \setlength\absrightindent{0cm}
  
  %fonte do ambiente%
  \abstracttextfont{\normalfont\normalsize}

  %intentação e espaçamento entre parágrafos%
  \setlength{\absparindent}{0pt}
  \setlength{\absparsep}{18pt}

%\pdfbookmark[0]{\listfigurename}{lof}
%\listoffigures*
%\cleardoublepage

\textual

\renewcommand{\ABNTEXchapterfont}{\fontfamily{cmr}\fontseries{b}\selectfont}
\renewcommand{\ABNTEXchapterfontsize}{\Large}

\renewcommand{\ABNTEXsectionfont}{\uppercase{\fontfamily{cmr}\fontseries{b}\selectfont}}
\renewcommand{\ABNTEXsectionfontsize}{\large}


\chapter{Introdução}
Está pesquisa procura aplicar um conjunto de dados musicais em um algortimo capaz de aprender a reconhecer padrões musicais a fim de identificar os covers adjacentes da música original. Na música popular, uma canção cover ou versão cover é definida como uma nova gravação produzida por alguém que não é um compositor ou cantor original, e essas canções cover compartilham elementos musicais importantes, como contornos da melodia, progressões harmônicas básicas e letras, com a canção original, que podem, e geralmente diferem da música original em outros aspectos, como instrumentação, andamento, ritmo, tom, harmonização e arranjo \cite{Chang2017}.Recuperar covers a partir de uma música original pode ser útil para auxiliar na estruturação de conteúdo, onde músicas similares e/ou adjacentes a uma música principal, podem ser melhor catalogadas/classificadas, além disso, algoritmos de sugestão de música podem ser mais rápidos quando a base de dados está melhor organizada. O objetivo do presente trabalho é aplicar machine learning para aprender padrões que identifiquem uma música como única e encontre seus covers, com uma taxa de assertividade que se iguale ou supere a do estado da arte, desprezando conceitos de desempenho. Inicialmente o artigo percorrerá pelos métodos atuais comumente usados para tal objetivo, e dissecara as caracteristicas usadas pelo aprendizado de máquina para aprender e categorizar a música e por fim mostrara os resultados do nosso objetivo.

%%Ler artigo que fale sobre a eficiência de base de dados mais estruturadas para busca mais rápida

%%Falta falar dos principais métodos usados e estudos na área

\end{itemize}
\chapter{Problemática da pesquisa e metodologia}

Tema = Extração de padrões musicais para encontrar emoções usando 

Problema = As ferramentas de reprodução de audio (player de música) não levam em consideração as emoções de seus usuarios.

Hipótese = Ferramentas de reprodução de audio que levam em consideração o estado emocional do usuario por meio de perguntas podem ser efetivas em detectar emoções de estresse.

Metodologia = Pesquisa bibliográfica e uma pequena coleta de dados para processamento e cruzamento de dados afim de reconhecer emoções de estresse.

Justificativa = Detectar emoções de estresse a partir de musicas pode ser útil para melhorar o algoritmo de sugestão musical com base no que o usuário sente.

%Descrição do problema de pesquisa a ser abordado, hipótese e pré-evidências (tanto do problema quanto para a hipótese).

%Metodologia a ser utilizada para verificar a hipótese, com justificativa.

\chapter{Resultados esperados}

Dissertar sobre os desdobramentos dos possíveis resultados do teste de sua hipótese. Pré-análise.

\section{Limitações do trabalho}

Dada as questões de pesquisa e a metodologia, as vezes é necessário clarificar que algumas dessas questões não podem ou não serão respondidas. Em outros casos, há questões muito próximas às abordadas, é útil clarificar e explicar porque ficaram de fora.

\chapter{Estado da arte}

Reconhecimento de cover é uma subárea do MIR (music information retrieval) e, de maneira grosseira, pode ser resumido em duas fases cruciais, a extração de recursos relevantes do áudio, e manipulação correta dos recursos extraídos, na respectiva ordem colocada. 

O sinal de áudio é frequentemente chamado de áudio bruto, em comparação com outras representações que são transformações baseadas nele \cite{Choi2018}, deste, as informações do áudio podem ser extraídas. A maioria das abordagens de aprendizado profundo em MIR tira proveito de representações bidimensionais em vez da representação unidimensional original que é o sinal de áudio bruto, e em muitos casos, as duas dimensões são eixos de frequência e tempo \cite{Choi2018}.

A fase de manipular os recursos extraídos em prol do reconhecimento de cover pode variar em vários fatores de acordo com a abordagem adotada. Um dos fatores importantes no reconhecimento de cover é a métrica de distância aplicada. Uma métrica de distância mede a similaridade de subsequências no espaço de recurso dentro de duas peças musicais \cite{Chang2017} diferentes ou não. 

No processo de reconhecer covers CHANG et al. primeiro converte os sinais de áudio de cada música em recursos de croma de 12 dimensões com uma janela não sobreposta de 1 segundo (fase de extração), usa como métrica de distância a matriz de similaridade cruzada e aplica CNN nas matrizes adjacentes geradas das comparações musicais. CHANG et al. supõe que uma rede neural possa identificar e aprender padrões inerentes da música original que não se perdem no cover \cite{Chang2017}. As matrizes de similaridade requerem espaço quadrático em relação ao comprimento do vetor de recursos usado para descrever o áudio. Por esse motivo, a maioria dos métodos usados para encontrar padrões na matriz de similaridade são (pelo menos) quadráticos em complexidade de tempo \cite{8392419}. Para melhorar o desempenho do processo de reconhecimento de cover, SILVA et al. propõe uma simplificação dessa matriz pois o mesmo acredita que a maioria das informações contidas em matrizes de similaridade seja irrelevante ou com pouco impacto em sua análise \cite{8392419}.Tomando essa ideia como base SILVA et al. propõe o SiMPle (Similarity Matrix ProfiLE), uma versão otimizada das matrizes de similaridade. Os recursos de croma foram usados, mas segundo o artigo, diferentes conjuntos de recursos podem ser utilizados. Para calcular o SIMPle, SILVA et al. sugere o uso do SIMPLe-fast, e posteriormente a média do SIMPLe gerado é usada para ligar o cover a música. SERRÀ, J. et al. também usa recursos de croma para comparar músicas diferentes. Inicialmente se extrai séries temporais do croma que representam sua progressão tonal \cite{Serra2009}. Essas séries temporais são então usadas para incorporação multivariada por meio de coordenadas de atraso \cite{Serra2009}. Para avaliar equivalências de estados entre os dois sistemas obtidos em momentos diferentes, foram usados CRP (Cross Recurrence Plot) e medidas de quantificação de recorrência derivadas deles \cite{Serra2009}. Na pré-análise, as medidas de quantificação de recorrência existentes foram avaliadas usando o KNN (k-nearest neighbors algorithm) \cite{Serra2009} e métricas de precisão no padrão IR (Information Retrieval), como o MAP (Mean Average Precision).

CHENG, Y et al. tenta uma abordagem diferente, ele propõe extração de recursos de arquivos MIDI (Musical Instrument Digital Interface). MIDI é um padrão técnico que permite uma ampla variedade de instrumentos musicais eletrônicos se conectarem, e se comunicarem entre si, e que também fornece uma forma de representação simbólica para a música \cite{10.1145/3077136.3080680}. O MIDI é outra fonte de extração de recursos para o reconhecimento de cover. O processo de extração de pitch é mais preciso em arquivos MIDI originais do que em arquivos de áudio, no entanto, arquivos MIDI originais geralmente são difíceis de serem acessados por motivos de direitos autorais \cite{10.1145/3077136.3080680}. Para extrair os recursos, primeiramente CHENG, Y et al. converte o sinal de áudio em MIDI, e nesse processo algumas informações relevantes podem ser perdidas \cite{10.1145/3077136.3080680}. Os recursos extraídos são chamados de NCP, a métrica de precisão adotada é o MAP (Mean Average Precision) e o algoritmo Q * MAX para tomar as decisões. 

PONIGHZWA, R. M. F. et al., busca seus recursos no padrão MPEG-7 e propõe usar 2 recursos de extração: a projeção do espectro de áudio e o nivelamento do espectro de áudio \cite{8257086}. Para o reconhecimento de cover é usado o algoritimo KNN (k-nearest neighbors algorithm) modificado e a métrica de acurácia é uma modificação aparente, do MAP (Mean Average Precision).



\chapter{Cronograma}

Faça um cronograma de atividades por semana de trabalho.
\begin{itemize}
\item Somente atividades da metodologia e etapas da escrita da monografia.
\item Não deve repetir as atividades da disciplina de TCC.
\item É necessário ser definido junto ao orientador.
\end{itemize}

\chapter{Leitura e fichamento da bibliografia}

\section{Fast Similarity Matrix Profile for Music Analysis and Exploration}

\begin{description}
\item[Autores:] Silva, Diego F. and Yeh, Chin-Chia M. and Zhu, Yan and Batista, Gustavo E. A. P. A. and Keogh, Eamonn
\item[Ano:] 2019
\item[Relevância:] 13 Citações (Google Acadêmico)
\end{description}

{\bfseries Dicionário:}

\item Matriz de semelhança = Uma representação em duas dimensões da semelhança entre duas matrizes diferentes.
\item Chroma Energy Normalized Statistics (CENS) = O CENS é uma representação derivada da energia de Chroma, que é gerada após algumas etapas de pós-processamento, no intuito de reduzir ao máximo as discrepâncias não essenciais de uma música, a fim de extrair apenas sua sequência de notas.
\item Pearson Correlation Coefficient (PCC) = Coeficiente usado para medir a correlação entre duas variáveis distintas.

{\bfseries Palavras-chave:}

\item MASS - Algoritmo mais eficiente conhecido para calcular vetor de distância.
\item DTW

{\bfseries Resenha}

Uma das maneiras para comparar, não só canções, mas outros tipos de arquivo, é a criação de uma matriz de semelhança, composta por 2 dimensões. Esse tipo de abordagem requer um espaço quadrático em relação ao comprimento do vetor de recursos usados para descrever os áudios contrapostos, fator que torna o processamento quadrático em complexidade de tempo. Segundo o artigo, esse tamanho expressivo tem sua causa nas demasiadas informações, julgadas desnecessárias pelos autores, que abrem possibilidade de otimização se assim o for. O SIMPLE-fast é uma maneira de extrair a matriz de similaridade, apenas com as informações essenciais, em um tempo reduzido.
Para fazer está comparação, primeiramente o áudio é separado em recortes temporais, onde cada um desses recortes é descrito adentro de um vetor. Cada vetor armazena informações do respectivo recorte que representa (o SIMPLe-fast pode se adaptar a diversas representações de áudio). A extração do SIMPLe, basicamente se forma sobre o cálculo de vetor de distância. Utilizando o MASS, o cálculo de vetor de distância pode ser melhorado, pois não é necessário calcular do zero a cada iteração, já que os resultados do cálculo da janela anterior podem ser reutilizados. 

Mudanças na ordem estrutural da música, podem inserir e excluir segmentos SIMPLe é resistente a variações estruturais, e/ou adições/exclusões em uma música. O artigo mostra uma tentativa interessante de unir o reconhecimento de cover a partir de uma entrada fracionada, como em um streaming, usando o SIMPLe e SIMPLe-fast para atribuir em tempo real qual é o correspondente da música reproduzida.

\section{Audio Cover Song Identification using Convolutional Neural Network}

\begin{description}
\item[Autores:] Chang, Sungkyun and Lee, Juheon and Choe, Sang Keun and Lee, Kyogu
\item[Ano:] 2017
\item[Relevância:] 22 citações (Google Acadêmico)
\end{description}

{\bfseries Dicionário:}

\item Convolutional Neural Network = Rede Neural Artificial.

\item MAP = Maneira de avaliar os acertos totais do algoritmo de reconhecimento.


{\bfseries Palavras-chave:}

\item CNN

{\bfseries Resenha}

O artigo propõe o uso de uma rede neural para analisar e comparar a semelhança dos covers. O sistema proposto consiste em três etapas. No estágio de pré-processamento, convertemos os sinais de áudio em recursos de croma para cada música. Em seguida, geramos matrizes de similaridade cruzada tomando um par de recursos de croma. Por fim, o sistema já treinado, faz a classificação na saída do CNN.


\section{Cross recurrence quantification for cover song identification}

\begin{description}
\item[Autores:] Joan Serrà and Xavier Serra and Ralph G Andrzejak
\item[Ano:] 2009
\item[Relevância:] 206 citações (Google Acadêmico)
\end{description}
{\bfseries Dicionário:}

\item Mid-level feature = A sequência tonal da música. Pode ser a sequência de notas, ou a sequência de acordes.

{\bfseries Palavras-chave:}

\item Pitch Class Profile = vetores de recursos que representam a intensidade de cada um dos doze semitons da escala tonal.

\item Cross Recurrence Plot (CRP) = Um gráfico (ou matriz), que mostra todos os momentos em que um estado de um sistema dinâmico ocorre simultaneamente com um segundo sistema dinâmico. No caso do MIR, o CRP é usado para comparar as séries temporais de duas músicas diferentes.

{\bfseries Resenha}
Os métodos de identificação de Cover procuram extrair a sequência tonal para usar na comparação de músicas, esse recurso extraído geralmente é o chroma. A ideia de encontrar as músicas subjacentes de outra música, é uma tarefa de comparação interessante, pois deduz que cada música possui uma essência, que por sua vez, pode ser encontrada e rastreada. 

\section{Effective Music Feature NCP: Enhancing Cover Song Recognition with Music Transcription}

\begin{description}
\item[Autores:] Cheng, Yao and Chen, Xiaoou and Yang, Deshun and Xu, Xiaoshuo
\item[Ano:] 2017
\item[Relevância:] 5 citações (Google Acadêmico)
\end{description}

{\bfseries Dicionário:}

\item Beat-Chroma = É o recurso chroma que é invariável a variação do tempo/ritmo.

{\bfseries Palavras-chave:}


{\bfseries Resenha}

Os métodos de reconhecimento de cover, devem encontrar um método, ou recurso que represente a música e satisfaça a invariância de tonalidade, invariância de ritmo e invariância de estrutura. Os MIDI originais da música, são de difícil obtenção. Os recursos MIDI ainda não são competitivos as outras representações no campo de reconhecimento de cover, pois são recursos grandes e longos.

\section{A Tutorial on Deep Learning for Music Information Retrieval}

\begin{description}
\item[Autores:] Choi, Keunwoo and Fazekas, György and Cho, Kyunghyun and Sandler, Mark
\item[Ano:] 2018
\item[Relevância:] 66 citações (Google Acadêmico)
\end{description}

{\bfseries Dicionário:}

\item camada neural = conjunto de valores escalares (nós), que se moldam para aprender.

{\bfseries Palavras-chave:}
 
 \item CQT = Representação de áudio

{\bfseries Resenha}

As redes neurais profundas possuem mais camadas neurais que as redes neurais simples. Uma abordagem com rede neural profunda, geralmente é assumida quando a quantidade de dados é elevada. O Espectrograma Mel é uma representação sonora otimizada para a percepção humana.

\section{Cover song recognition based on MPEG-7 audio features}

\begin{description}
\item[Autores:] Ponighzwa, R. Mochammad Faris and Sarno, Riyanarto and Sunaryono, Dwi
\item[Ano:] 2017
\item[Relevância:] 1 citação (Google Acadêmico)
\end{description}

{\bfseries Dicionário:}

\item KNN = Algoritmo de machine learning, usado para classificar os dados.

{\bfseries Palavras-chave:}


{\bfseries Resenha}

É possível usar formatos de compressão, como representações de áudio. O MPEG7 possui diversos recursos atrelados a ele que podem ser extraídos e usados na tarefa de reconhecimento de cover. A biblioteca Java MPEG7AudioEncApp recebe uma música com extensão '.wav' como entrada, e a saída é um documento com extensão '.xml'. Discrete Wavelet Transform ajuda a eliminar o ruído de um espectrograma de instrumento não dominante (em geral, qualquer instrumento que não seja a voz).



\cleardoublepage

\postextual

%%Colocar as referências conforme as normas da ABNT, somente as utilizadas no trabalho e presentes neste manuscrito.

\bibliography{bibliografia}{}

% \begin{thebibliography}{99}

% \bibitem{ABNTEX2:2014}
% {ABNTEX2; ARAUJO, L. C. \textbf{A classe abntex2}: Documentos técnicos e científicos brasileiros compatíveis com as normas ABNT. Sine loco, v. 1.9.2; 2014.}.

% \bibitem{Biazin:2008}
% {BIAZIN, D. T. \textbf{Normas da ABNT e padronização de trabalhos acadêmicos}. Londrina: Instituto Filadélfia de Londrina; 2008.}

% \bibitem{Buneman:2011}
% {BUNEMAN, P.; CHENEY, J.; LINDLEY, S. et al. \textbf{DBWiki}: A Structured Wiki for Curated Data and Collaborative Data Management. Athens: SIGMOD’11; 2011.}

% \bibitem{Wikibooks:2014}
% {WIKIBOOKS. \textbf{LaTeX}: The Free Textbook Project. Disponível em: <http://en.wikibooks.org/wiki/LaTeX>. Acesso em: 09 abr. 2014.}

% \end{thebibliography}

\end{document}