%Para melhor utilização da ferramenta, utilize o pdfLaTeX+MakeIndex+BibTeX%

\documentclass[12pt,openright,oneside,a4paper,english,french,spanish,brazil]{unifil}

\titulo{Uso de machine learning para reconhecimento de cover} % Título da monografia
\autor{Gabriel Próspero Realizador Silva} % Nome do autor
\instituicao{Centro Universitário Filadélfia}
\local{Londrina}
\data{2021} % Ano de publicação
\preambulo{Engenharia de Software} % Nome do curso
\orientador{Mario Henrique Adaniya} % Nome do orientador

\begin{document}

\frenchspacing

%%%%%%%%%%%%%%%%%%%%%%%%%%%%
%% Elementos pré-textuais %%
%%%%%%%%%%%%%%%%%%%%%%%%%%%%

\pretextual

\imprimircapa
\makeatletter
\renewcommand{\folhaderostocontent}{
\begin{center}
{\ABNTEXchapterfont\bfseries\MakeTextUppercase{\imprimirautor}}
\vspace*{5cm}
\begin{center}
\ABNTEXchapterfont\bfseries\normalsize\MakeTextUppercase{\imprimirtitulo}
\end{center}
\vspace*{1cm}
\abntex@ifnotempty{\imprimirpreambulo}{%
\hspace{.45\textwidth}
\begin{minipage}{.5\textwidth}
\SingleSpacing
 Trabalho de Dissertação apresentado ao \imprimirinstituicao~como parte dos requisitos para obtenção de graduação em \imprimirpreambulo.
{\textnormal{\imprimirorientadorRotulo~\imprimirorientador.}}
\end{minipage}%
}%
\vspace*{\fill}
{\bfseries\large\imprimirlocal}
\par
{\bfseries\large\imprimirdata}
\vspace*{1cm}
\end{center}
}
\makeatother

\imprimirfolhaderosto

\clearpage{\pagestyle{empty}\cleardoublepage}
%%Altere as informações do resumo%%
\noindent{SOBRENOME; NOME, D. \textbf{\imprimirtitulo}. Trabalho de Conclusão de Curso (Graduação) - \imprimirinstituicao. \imprimirlocal, \imprimirdata.}
\par
\begin{resumo}
	%%Escreva seu resumo na língua vernácula aqui%%
\lipsum[5]
\vspace{\onelineskip} 
	%%Adicione as palavras chaves após os dois pontos '':''%%
\noindent
\textbf{Palavras-chaves}: 3 ou mais.

\end{resumo}


%\par
%\vspace{11cm}

\tableofcontents*

  \setlength\absleftindent{0cm}
  \setlength\absrightindent{0cm}
  
  %fonte do ambiente%
  \abstracttextfont{\normalfont\normalsize}

  %intentação e espaçamento entre parágrafos%
  \setlength{\absparindent}{0pt}
  \setlength{\absparsep}{18pt}

%\pdfbookmark[0]{\listfigurename}{lof}
%\listoffigures*
%\cleardoublepage

\textual

\renewcommand{\ABNTEXchapterfont}{\fontfamily{cmr}\fontseries{b}\selectfont}
\renewcommand{\ABNTEXchapterfontsize}{\Large}

\renewcommand{\ABNTEXsectionfont}{\uppercase{\fontfamily{cmr}\fontseries{b}\selectfont}}
\renewcommand{\ABNTEXsectionfontsize}{\large}


\chapter{Introdução}
Está pesquisa procura aplicar um conjunto de dados musicais em um algoritmo capaz de aprender a reconhecer padrões musicais a fim de identificar os covers adjacentes da música original. O cover nada mais é que uma interpretação de uma musica feita por um cantor que não o autor/proprietário da canção. Recuperar covers a partir de uma música original pode ser útil para auxiliar na estruturação de conteúdo, onde músicas similares e/ou adjacentes a uma música principal, podem ser melhor catalogadas/classificadas. O objetivo do presente trabalho é aplicar KNN (K — Nearest Neighbors) para aprender padrões que identifiquem uma música como única e encontre seus covers, com uma taxa de assertividade que se iguale ou supere a do estado da arte. Inicialmente o artigo percorrerá pelos métodos atuais comumente usados para tal objetivo, e dissecara as características usadas pelo aprendizado de máquina para aprender e categorizar a música, e por fim mostrara os resultados da pesquisa.

\end{itemize}
\chapter{Problemática da pesquisa e metodologia}

Tema = Extração de padrões musicais para encontrar emoções usando 

Problema = As ferramentas de reprodução de audio (player de música) não levam em consideração as emoções de seus usuarios.

Hipótese = Ferramentas de reprodução de audio que levam em consideração o estado emocional do usuario por meio de perguntas podem ser efetivas em detectar emoções de estresse.

Metodologia = Pesquisa bibliográfica e uma pequena coleta de dados para processamento e cruzamento de dados afim de reconhecer emoções de estresse.

Justificativa = Detectar emoções de estresse a partir de musicas pode ser útil para melhorar o algoritmo de sugestão musical com base no que o usuário sente.

%Descrição do problema de pesquisa a ser abordado, hipótese e pré-evidências (tanto do problema quanto para a hipótese).

%Metodologia a ser utilizada para verificar a hipótese, com justificativa.

\chapter{Resultados esperados}

Esse trabalho espera estudar algumas técnicas de extração de recursos dos sinais de áudio bruto, algoritmos e técnicas para comparação desses recursos e técnicas de machine learning para classificação/agrupamento de músicas. Se a pesquisa for produtiva o suficiente, este trabalho visa entregar um algoritmo KNN (K — Nearest Neighbors) modificado que seja capaz de agrupar covers com sua determinada música original.

\section{Limitações do trabalho}

Este trabalho não visa lidar com conjuntos de dados massivos, problema de gerenciamento e processamento que também faz parte do estado da arte mas que demandaria de mais tempo e conhecimento profundo sobre otimização de recursos e outras áreas não revisadas aqui. 

\chapter{Estado da arte}

Reconhecimento de cover é uma subárea do \textit{Music Information Retrieval} (MIR) e, de maneira grosseira, pode ser resumido em duas fases cruciais, a extração de recursos relevantes do áudio, e manipulação correta dos recursos extraídos, na respectiva ordem colocada. 

O sinal de áudio é frequentemente chamado de áudio bruto, em comparação com outras representações que são transformações baseadas nele \cite{Choi2018}, deste, as informações do áudio podem ser extraídas. A maioria das abordagens de aprendizado profundo em MIR tira proveito de representações bidimensionais em vez da representação unidimensional original que é o sinal de áudio bruto, e em muitos casos, as duas dimensões são eixos de frequência e tempo \cite{Choi2018}, essas representações são chamadas de recursos de áudio

A fase de manipular os recursos extraídos em prol do reconhecimento de cover pode variar em vários fatores de acordo com a abordagem adotada. Um dos fatores importantes no reconhecimento de cover é a métrica de distância aplicada. Uma métrica de distância mede a similaridade de subsequências no espaço de recurso dentro de duas peças musicais \cite{Chang2017} diferentes ou não. 

No processo de reconhecer covers CHANG et al. primeiro converte os sinais de áudio de cada música em recursos de croma de 12 dimensões com uma janela não sobreposta de 1 segundo (fase de extração), usa como métrica de distância a matriz de similaridade cruzada e aplica CNN nas matrizes adjacentes geradas das comparações musicais. CHANG et al. supõe que uma rede neural possa identificar e aprender padrões inerentes da música original que não se perdem no cover \cite{Chang2017}. As matrizes de similaridade requerem espaço quadrático em relação ao comprimento do vetor de recursos usado para descrever o áudio. Por esse motivo, a maioria dos métodos usados para encontrar padrões na matriz de similaridade são (pelo menos) quadráticos em complexidade de tempo \cite{8392419}. Para melhorar o desempenho do processo de reconhecimento de cover, SILVA et al. propõe uma simplificação dessa matriz pois o mesmo acredita que a maioria das informações contidas em matrizes de similaridade seja irrelevante ou com pouco impacto em sua análise \cite{8392419}.Tomando essa ideia como base SILVA et al. propõe o SiMPle (Similarity Matrix ProfiLE), uma versão otimizada das matrizes de similaridade. Os recursos de croma foram usados, mas segundo o artigo, diferentes conjuntos de recursos podem ser utilizados. Para calcular o SIMPle, SILVA et al. sugere o uso do SIMPLe-fast, e posteriormente a média do SIMPLe gerado é usada para ligar o cover a música. SERRÀ, J. et al. também usa recursos de croma para comparar músicas diferentes. Inicialmente se extrai séries temporais do croma que representam sua progressão tonal \cite{Serra2009}. Essas séries temporais são então usadas para incorporação multivariada por meio de coordenadas de atraso \cite{Serra2009}. Para avaliar equivalências de estados entre os dois sistemas obtidos em momentos diferentes, foram usados CRP (Cross Recurrence Plot) e medidas de quantificação de recorrência derivadas deles \cite{Serra2009}. Na pré-análise, as medidas de quantificação de recorrência existentes foram avaliadas usando o KNN (k-nearest neighbors algorithm) \cite{Serra2009} e métricas de precisão no padrão IR (Information Retrieval), como o MAP (Mean Average Precision).

CHENG, Y et al. tenta uma abordagem diferente, ele propõe extração de recursos de arquivos MIDI (Musical Instrument Digital Interface). MIDI é um padrão técnico que permite uma ampla variedade de instrumentos musicais eletrônicos se conectarem, e se comunicarem entre si, e que também fornece uma forma de representação simbólica para a música \cite{10.1145/3077136.3080680}. O MIDI é outra fonte de extração de recursos para o reconhecimento de cover. O processo de extração de pitch é mais preciso em arquivos MIDI originais do que em arquivos de áudio, no entanto, arquivos MIDI originais geralmente são difíceis de serem acessados por motivos de direitos autorais \cite{10.1145/3077136.3080680}. Para extrair os recursos, primeiramente CHENG, Y et al. converte o sinal de áudio em MIDI, e nesse processo algumas informações relevantes podem ser perdidas \cite{10.1145/3077136.3080680}. Os recursos extraídos são chamados de NCP, a métrica de precisão adotada é o MAP (Mean Average Precision) e o algoritmo Q * MAX para tomar as decisões. 

PONIGHZWA, R. M. F. et al., busca seus recursos no padrão MPEG-7 e propõe usar 2 recursos de extração: a projeção do espectro de áudio e o nivelamento do espectro de áudio \cite{8257086}. Para o reconhecimento de cover é usado o algoritimo KNN (k-nearest neighbors algorithm) modificado e a métrica de acurácia é uma modificação aparente, do MAP (Mean Average Precision).



\chapter{Cronograma}

Faça um cronograma de atividades por semana de trabalho.
\begin{itemize}
\item Somente atividades da metodologia e etapas da escrita da monografia.
\item Não deve repetir as atividades da disciplina de TCC.
\item É necessário ser definido junto ao orientador.
\end{itemize}

\chapter{Leitura e fichamento da bibliografia}

Neste capítulo, registrem todo o trabalho de fichamento de bibliografia, utilizando a seguinte estrutura.

\section{Título da obra bibliográfica 1}

\begin{description}
\item[Autores:] Fulano \emph{et al.}
\item[Ano:] 2017
\item[Relevância:] Qualis, índice $H^*$ ou quantidade de citações.
\end{description}

Faça um dicionário, anote todos os termos desconhecidos seguidos de seus significados recém aprendidos.

Faça uma lista de palavras-chave descobertas neste artigo e que foram relevantes para novas buscas de levantamento bibliográfico.

Escreva uma resenha em prosa daquilo que você apreendeu da obra em questão.

\section{Título da obra bibliográfica 2}

\begin{description}
\item[Autores:] Ciclano \emph{et al.}
\item[Ano:] 2015
\item[Relevância:] Qualis, índice $H^*$ ou quantidade de citações.
\end{description}

Faça um dicionário, anote todos os termos desconhecidos seguidos de seus significados recém aprendidos.

Faça uma lista de palavras-chave descobertas neste artigo e que foram relevantes para novas buscas de levantamento bibliográfico.

Escreva uma resenha em prosa daquilo que você apreendeu da obra em questão.

\cleardoublepage

\postextual

%%Colocar as referências conforme as normas da ABNT, somente as utilizadas no trabalho e presentes neste manuscrito.

\bibliography{bibliografia}{}

% \begin{thebibliography}{99}

% \bibitem{ABNTEX2:2014}
% {ABNTEX2; ARAUJO, L. C. \textbf{A classe abntex2}: Documentos técnicos e científicos brasileiros compatíveis com as normas ABNT. Sine loco, v. 1.9.2; 2014.}.

% \bibitem{Biazin:2008}
% {BIAZIN, D. T. \textbf{Normas da ABNT e padronização de trabalhos acadêmicos}. Londrina: Instituto Filadélfia de Londrina; 2008.}

% \bibitem{Buneman:2011}
% {BUNEMAN, P.; CHENEY, J.; LINDLEY, S. et al. \textbf{DBWiki}: A Structured Wiki for Curated Data and Collaborative Data Management. Athens: SIGMOD’11; 2011.}

% \bibitem{Wikibooks:2014}
% {WIKIBOOKS. \textbf{LaTeX}: The Free Textbook Project. Disponível em: <http://en.wikibooks.org/wiki/LaTeX>. Acesso em: 09 abr. 2014.}

% \end{thebibliography}

\end{document}