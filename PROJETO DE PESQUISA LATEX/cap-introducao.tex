
\chapter{Introdução}
Está pesquisa procura aplicar um conjunto de dados musicais em um algoritmo capaz de aprender a reconhecer padrões musicais a fim de identificar os covers adjacentes da música original. O cover nada mais é que uma interpretação de uma musica feita por um cantor que não o autor/proprietário da canção. Recuperar covers a partir de uma música original pode ser útil para auxiliar na estruturação de conteúdo, onde músicas similares e/ou adjacentes a uma música principal, podem ser melhor catalogadas/classificadas. O objetivo do presente trabalho é aplicar KNN (K — Nearest Neighbors) para aprender padrões que identifiquem uma música como única e encontre seus covers, com uma taxa de assertividade que se iguale ou supere a do estado da arte. Inicialmente o artigo percorrerá pelos métodos atuais comumente usados para tal objetivo, e dissecara as características usadas pelo aprendizado de máquina para aprender e categorizar a música, e por fim mostrara os resultados da pesquisa.

\end{itemize}