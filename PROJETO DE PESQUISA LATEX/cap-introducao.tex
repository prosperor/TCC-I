
\chapter{Introdução}
Está pesquisa procura aplicar um conjunto de dados musicais em um algortimo capaz de aprender a reconhecer padrões musicais a fim de identificar os covers adjacentes da música original. Na música popular, uma canção cover ou versão cover é definida como uma nova gravação produzida por alguém que não é um compositor ou cantor original, e essas canções cover compartilham elementos musicais importantes, como contornos da melodia, progressões harmônicas básicas e letras, com a canção original, que podem, e geralmente diferem da música original em outros aspectos, como instrumentação, andamento, ritmo, tom, harmonização e arranjo \cite{Chang2017}.Recuperar covers a partir de uma música original pode ser útil para auxiliar na estruturação de conteúdo, onde músicas similares e/ou adjacentes a uma música principal, podem ser melhor catalogadas/classificadas, além disso, algoritmos de sugestão de música podem ser mais rápidos quando a base de dados está melhor organizada. O objetivo do presente trabalho é aplicar machine learning para aprender padrões que identifiquem uma música como única e encontre seus covers, com uma taxa de assertividade que se iguale ou supere a do estado da arte, desprezando conceitos de desempenho. Inicialmente o artigo percorrerá pelos métodos atuais comumente usados para tal objetivo, e dissecara as caracteristicas usadas pelo aprendizado de máquina para aprender e categorizar a música e por fim mostrara os resultados do nosso objetivo.

%%Ler artigo que fale sobre a eficiência de base de dados mais estruturadas para busca mais rápida

%%Falta falar dos principais métodos usados e estudos na área

\end{itemize}