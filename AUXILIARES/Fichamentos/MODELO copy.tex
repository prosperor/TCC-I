\documentclass{article}
\usepackage[utf8]{inputenc}
\title{Fichamento}
\author{GABRIEL PRÓSPERO REALIZADOR  SILVA}
\date{Março 2021}

\usepackage{natbib}
\usepackage{graphicx}
\begin{document}
\maketitle

\section{DADOS DO ARTIGO}
\textbf{Methodology for Stage Lighting Control Based on Music Emotions \\}
\author{AUTOR \\}
\date{DATA EM QUE O ARTIGO FOI FEITO}

\section{RESENHA}
é difícil usar um pequeno número de adjetivos para cobrir todas as expressões possíveis de emoção musical. Além disso, pessoas com origens culturais diferentes interpretam os adjetivos de maneira diferente. Portanto, os pesquisadores começaram a buscar um meio geral e eficiente de expressar as emoções musicais. Um modelo de emoção, conhecido como modelo de Thayer, que se baseia em expressões de emoção lineares e contínuas, agora é usado com frequência.


\section{PLAVRAS-CHAVES}
\begin{itemize}
    \item music emotion recognition (MER)
    \item Thayer’s emotion plane
    \item support vector regression (SVR)
    \item Kansei Engineering measurement approach
    \item linear prediction coefficients (LPC)
    \item linear prediction cepstrum coefficients (LPCC)
    \item mel-frequency cepstrum coefficients (MFCC)
    \item cepstros
    \item roll-off
\end{itemize}

\end{document}

