\documentclass{article}
\usepackage[utf8]{inputenc}
\title{Fichamento}
\author{GABRIEL PRÓSPERO REALIZADOR  SILVA}
\date{Março 2021}

\usepackage{natbib}
\usepackage{graphicx}
\begin{document}
\maketitle

\section{DADOS DO ARTIGO}
\textbf{Novel Audio Features for Music Emotion Recognition \\}
\author{Renato Panda \\}
\date{2020}

\section{RESENHA}
Freqüentemente, argumenta-se que os paradigmas dimensionais levam a uma menor ambiguidade, uma vez que, em vez de ter um conjunto discreto de adjetivos de emoção, as emoções são consideradas um continuum [10] Um modelo dimensional amplamente aceito em MER é o modelo circunplexo de James Russell [13].
As duas dimensões propostas são valência (agradável-desagradável) e atividade ou excitação (excitado-não excitado), ou AV. O plano bidimensional resultante forma quatro quadrantes diferentes: 1- exuberância, 2- ansiedade, 3- depressão e 4- contentamento (Fig. 1).
Ainda assim, acreditamos que estimar coisas como linhas melódicas predominantes, mesmo que imperfeitas, nos dá informações relevantes que atualmente não são utilizadas no MER. Para tanto, nos baseamos em trabalhos anteriores de Salomon et al. [38] e Dressler [39] para estimar as frequências fundamentais predominantes (f0) e saliências. Normalmente, o processo começa identificando quais frequências estão presentes no sinal em cada ponto no tempo (extração da sinusóide).


\section{PLAVRAS-CHAVES}
\begin{itemize}
    \item Music Information Retrieval (MIR)
    \item MIREX AMC
    \item Arousal, valence and dominance (AVD)
    \item extração da sinusóide
    \item função de saliência do tom
    \item 
\end{itemize}

\end{document}

