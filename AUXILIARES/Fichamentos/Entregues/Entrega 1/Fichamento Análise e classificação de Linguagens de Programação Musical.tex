\documentclass{article}
\usepackage[utf8]{inputenc}
\title{Fichamento}
\author{GABRIEL PRÓSPERO REALIZADOR  SILVA}
\date{Março 2021}
\usepackage{natbib}
\usepackage{graphicx}
\begin{document}
\maketitle

\section{DADOS DO ARTIGO}
\textbf{ANÁLISE E CLASSIFICAÇÃO DE LINGUAGENS DE PROGRAMAÇÃO MUSICAL \\}
\author{Rodrigo Ramos De Araujo \\}
\date{2018}

\section{RESENHA}
 A programação musical pode ser feita por meio de linguagens de propósito geral (GPL - General Purpose Language), como Java, Python ou C, que permitem o desenvolvimento de sistemas de software para esta finalidade, desde que sejam combinadas com recursos específicos para o domínio de computação musical, proporcionando ao ambiente responsável pela execução a capacidade de acessar os dispositivos de som do computador (SCHIAVONI; GOULART; QUEIROZ, 2012). 
 Outra possibilidade para o desenvolvimento de aplicações musicais é a utilização de linguagens específicas de domínio (DSL - Domain Specific Language) que são linguagens elaboradas com base nos conceitos voltados para uma área específica de atuação. 
 Diversos campos fazem uso constante de linguagens de programação musical, como: a área de processamento de sinais para música, engenharia de áudio, sonificação, produção musical, criação de novos instrumentos musicais, composição musical apoiada por computador, composição algorítmica, análise musical apoiada por computador, criação de instalação artísticas, sistemas multimídias, e realidade virtual. 
 Também é possível a análise de áudios para recuperação de informações musicais como: estilos; tempo; tonalidade; ornamentos; e afins (CUTHBERT; ARIZA, 2010). 
 Metodologia Goal/Question/Metric (GQM) (BASILI, 1992). 
 As linguagens interpretadas podem garantir um ambiente de execução que garanta a alta disponibilidade do processamento assim como o agendamento das tarefas dentro de um intervalo de tempo compatível com a taxa de amostragem do som gerado. 
 Tal garantia é menor para linguagens compiladas ou geradoras de código. 
 Neste caso, o programador precisa ter conhecimento para configurar seu sistema operacional para a execução em tempo real de maneira a garantir o processamento do áudio sem interrupção. 
 Por outro lado, linguagens compiladas atuam diretamente sobre o Sistema operacional e podem ter um custo mais leve de execução por não necessitar do ambiente de interpretação. 
 Por esta razão, ser compilada ou executada não pode ser considerado um atributo positivo ou negativo, mas como uma característica que pode ser explorada e avaliada no momento de optar por uma ou outra linguagem de programação. 

\section{PLAVRAS-CHAVES}
\begin{itemize}
    \item COMPUTAÇÃO MUSICAL
    \item LINGUAGENS DE PROGRAMAÇÃO
    \item LINGUAGENS ESPECÍFICAS DE DOMÍNIO
    \item GQM
\end{itemize}

\end{document}
