\documentclass{article}
\usepackage[utf8]{inputenc}
\title{Fichamento}
\author{GABRIEL PRÓSPERO REALIZADOR  SILVA}
\date{Maiol 2021}

\usepackage{natbib}
\usepackage{graphicx}
\begin{document}
\maketitle

\section{DADOS DO ARTIGO}
\textbf{Cross recurrence quantification for cover song identification \\}
\author{Joan Serr{\`{a}} and Xavier Serra and Ralph G Andrzejak\\}
\date{2009}

\section{RESENHA}
Métodos para identificação automática de cover song geralmente exploram similaridade de sequência tonal e tentam ser robustos contra mudanças comuns em outros aspectos musicais [18]. Em geral, eles buscam extrair a melodia predominante, uma progressão de acordes ou uma série temporal de croma (um recurso de nível médio que representa o conteúdo harmônico) do sinal de áudio bruto e torná-lo independente da tonalidade principal. Em seguida, para obter uma medida de similaridade entre as canções, as séries temporais do descritor de tonalidade são geralmente comparadas por meio de técnicas como sincronização temporal dinâmica, variantes de distância de edição, algoritmos de correspondência de string, hashing de subsequência ou por funções de similaridade comuns
De um ponto de vista prático e comercial, quantificar a similaridade musical é a chave para pesquisar e organizar coleções musicais automaticamente. Além disso, identificar canções cover tem uma implicação direta na gestão e licenças de direitos musicais. Além disso, do ponto de vista do usuário, encontrar todas as versões de uma determinada música pode ser valioso e divertido. Para estimar as sequências tonais de peças musicais, pode-se empregar recursos de croma ou pitch class profile (PCP). Eles são amplamente usados ​​na comunidade MIR [42] - [45] e comprovadamente funcionam bem como informações primárias para sistemas de identificação de covers [18].
A composição de conceitos dessas diferentes disciplinas, resulta naturalmente em uma organização modular do nosso método. Dados dois sinais de áudio, nós, a princípio, usamos técnicas de processamento de sinais musicais para extrair séries temporais de descritores que representam sua progressão tonal. Essas séries temporais são então usadas para incorporação multivariada por meio de coordenadas de atraso. Para avaliar equivalências de estados entre os dois sistemas obtidos em momentos diferentes, usamos CRPs e medidas de quantificação de recorrência derivadas deles. Na pré-análise, as medidas de quantificação de recorrência existentes foram avaliadas usando técnicas de aprendizado de máquina. O resultado obtido nos motivou a introduzir novas medidas de quantificação de recorrência cruzada Smax e Qmax.
\section{PLAVRAS-CHAVES}
\begin{itemize}
    \item Cover
    \item Aprendizado de Máquina
\end{itemize}

\end{document}

