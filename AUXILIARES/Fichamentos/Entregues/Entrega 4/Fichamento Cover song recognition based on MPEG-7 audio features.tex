\documentclass{article}
\usepackage[utf8]{inputenc}
\title{Fichamento}
\author{GABRIEL PRÓSPERO REALIZADOR  SILVA}
\date{Março 2021}

\usepackage{natbib}
\usepackage{graphicx}
\begin{document}
\maketitle

\section{DADOS DO ARTIGO}
\textbf{Cover Song Recognition Based on MPEG-7 Audio Features \\}
\author{Ponighzwa, R. Mochammad Faris and Sarno, Riyanarto and Sunaryono, Dwi \\}
\date{2017}

\section{RESENHA}
Basicamente, a música é um som / áudio que tem um tom. O tom não pode ser assistido diretamente, mas pode ser observado como espectrograma por meio do software como um sinal. 
A música pode ser identificada com base em sua parte específica do espectrograma. Este método só é adequado para identificar uma música gravada com seu original porque o espectrograma da música gravada deve ter a mesma parte do espectrograma de seu original. 
Este método não pode ser aplicado para identificar uma música cover em seu original, pois às vezes uma música cover tem o mesmo tom da música original, mas ainda parece "diferente" do original. 
Este sentido “diferente” ocorre porque o artista da capa cantou no mesmo tom que o artista original, mas tocou em uma nota diferente do artista original. Pode ser mais baixo ou mais alto da música original, dependendo da capacidade do artista cover. 
Outro problema com esse método é que às vezes há algumas músicas cover cantadas por um gênero específico, mas a música original foi cantada por um gênero oposto. 
Se isso acontecer, é muito improvável que as canções cover possam ser identificadas com sua canção original devido ao fato de que o espectrograma de masculino e feminino é diverso, embora eles cantem a mesma música. 
Aplicativos de reconhecimento de música, como Shazaam e Sound hound, usam um método de identificação baseado no espectrograma de uma música específica. 
Shazam e Sound hound usaram apenas uma certa parte do espectrograma, chamada impressão digital, e comparou sua impressão digital com seu banco de dados. 
Então, em outra palavra, eles não podem identificar uma música cover para seu original.
Para reconhecer o título de uma música original de uma música cover, é necessário comparar o espectrograma do cantor da música original e da versão. 
O espectrograma de um cantor pode ser recuperado aplicando o método wavelet a todo o espectrograma da música. Para recuperar o espectrograma de um vocal de uma música, ele precisa obter o método de passagem baixa do wavelet. O método de passagem baixa do wavelet é representado pelo valor de aproximação do espectrograma. Este valor de aproximação foi comparado com o valor de aproximação do conjunto de dados no banco de dados.
Quando o filtro passa-baixo é obtido do método wavelet, ele retorna o espectrograma que contém informações do artista vocal.
Para encontrar o nível de decomposição correto do método wavelet, aplicado (1) [9]. Primeiro, calcule o valor médio do espectrograma. Cada valor menos a média do espectrograma, a seguir, todos os valores absolutos e aplica o método da Transformação Rápida de Fourier (FFT). FFT é um método que transforma o espectrograma no domínio do tempo em espectrograma no domínio da frequência [10].
\section{PLAVRAS-CHAVES}
\begin{itemize}
    \item FFT
    \item COVER 
    \item MASCULINE VOCIE
    \item FEMININE VOICE
\end{itemize}

\end{document}

