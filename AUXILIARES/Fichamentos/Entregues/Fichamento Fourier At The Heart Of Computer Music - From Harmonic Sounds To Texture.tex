\documentclass{article}
\usepackage[utf8]{inputenc}
\title{Fichamento}
\author{GABRIEL PRÓSPERO REALIZADOR  SILVA}
\date{Março 2021}
\usepackage{natbib}
\usepackage{graphicx}
\begin{document}
\maketitle

\section{DADOS DO ARTIGO}
\textbf{FOURIER AT THE HEART OF COMPUTER MUSIC: FROM HARMONIC SOUNDS TO TEXTURE \\}
\author{VINCENT LOSTANLEN \\}
\date{2019}

\section{RESENHA}
A codificação de uma onda contínua de largura de banda B sem perda de informação requer amostras discretas de 2B por segundo. Na produção musical, uma ampla gama de efeitos de áudio digital (DAFX) depende de alguma forma de análise espectro temporal e re-síntese após manipulação no domínio do tempo-frequência. Um exemplo bem conhecido deste procedimento é a fase vocoder, no coração do algoritmo “Auto Tune” para ajuste de tom de voz. De forma mais geral, a questão de modelar texturas de áudio na ausência de qualquer conhecimento prévio de periodicidade temporal permanece amplamente aberta.

\section{PLAVRAS-CHAVES}
\begin{itemize}
    \item FOURIER ANALYSIS; 
    \item COMPUTER MUSIC; 
    \item AUDIO SIGNAL PROCESSING
\end{itemize}

\end{document}
