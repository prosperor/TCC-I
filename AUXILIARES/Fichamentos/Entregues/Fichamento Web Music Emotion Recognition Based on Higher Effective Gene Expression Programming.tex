\documentclass{article}
\usepackage[utf8]{inputenc}
\title{Fichamento}
\author{GABRIEL PRÓSPERO REALIZADOR  SILVA}
\date{Março 2021}
\usepackage{natbib}
\usepackage{graphicx}
\begin{document}
\maketitle

\section{DADOS DO ARTIGO}
\textbf{WEB MUSIC EMOTION RECOGNITION BASED ON HIGHER EFFECTIVE GENE EXPRESSION PROGRAMMING \\}
\author{KEJUN ZHANG; SHOUQIAN SUN \\}
\date{2012}

\section{RESENHA}
Reconhecimento de emoções musicais é uma subárea de estudo da Computação Musical. Yang et al propôs uma variedade hierárquica harmonizadora para recuperação de mídia cruzada, com uma nova estrutura para análise e recuperação de conteúdo multimídia com base em classificação semi-supervisionada e feedbacks de relevância. Os métodos estatísticos são usados como base principal para os testes do estudo. O SVM (Suport Vertors Machine) é avaliado como um dos melhores métodos estatísticos, mas perde em velocidade, também por isso a pesquisa busca encontrar um método mais eficiente no reconhecimento de emoções musicais. A proposta do trabalho é apresentar um algoritmo, chamado RGEP, dito como mais eficaz que o GEP e SVM, para fazer o reconhecimento das emoções musicais, Para fazer o experimento, foi usado um conjunto de dados de 600 partes rítmicas (clipes musicais de 30 segundos) no formato MP3 e 80 alunos que desconheciam as músicas usadas no experimento, e 8 classificações para as músicas. A emoção que tem a maior nota é a que define a música. No final do experimento o RGEP teve uma melhora em 15% com relação ao GEP e em aproximadamente 50% em relação ao SVM.

\section{PLAVRAS-CHAVES}
\begin{itemize}
    \item MUSIC EMOTION RECOGNITION; 
    \item EVOLUTIONARY ALGORITHM; 
    \item GENE EXPRESSION PROGRAMMING; 
    \item SUPPORT VECTOR MACHINE; 
    \item MUSIC INFORMATION RETRIEVAL;
\end{itemize}


\end{document}
