\documentclass{article}
\usepackage[utf8]{inputenc}
\title{Fichamento}
\author{GABRIEL PRÓSPERO REALIZADOR  SILVA}
\date{Abril 2021}

\usepackage{natbib}
\usepackage{graphicx}
\begin{document}
\maketitle

\section{DADOS DO ARTIGO}
\textbf{Leveraging Affective Hashtags for Ranking Music Recommendations \\}
\author{Eva Zangerle \\}
\date{2018}

\section{RESENHA}
RQ1: How can affective contextual information contribute to improving personalized ranking of track recommendation candidates?
RQ2: How can we computationally represent the affective contextual information in a #nowplaying tweet? 
Enquanto a primeira tarefa é principalmente sobre a preferência geral de um usuário (ou seja, quais faixas um usuário gostaria), a segunda tarefa requer a modelagem da preferência específica do contexto de um usuário, pois já sabemos que todas as faixas candidatas são apreciadas pelo usuário, mas apenas um deles pode ser classificado no topo, dado aquele contexto de escuta específico. 
Um algoritmo não pode funcionar bem se não souber como o contexto emocional de um usuário afeta sua preferência musical.
Por exemplo, um usuário pode ser representado pela própria representação latente do usuário (denotado como "usuário"), mas também pode ser representado pela representação latente média das hashtags que o usuário usou antes em seus tweets ("usertag"). 
Da mesma forma, uma trilha pode ser representada por sua própria representação latente (“trilha”) ou pela representação média das hashtags às quais a trilha foi associada por diferentes usuários (“tracktag”).
Além disso, o trabalho de Bollen et al. [44] sublinha essa escolha, pois os autores realizaram experimentos mostrando que apresentar aos usuários um grande número de itens bons e valiosos é contraproducente, pois a escolha de um item se torna inerentemente difícil para o usuário.
Para esta tarefa, contamos com um método de detecção de sentimento não supervisionado amplamente utilizado: o chamado sentiment lexica [19].
Goethem e Sloboda [3] descobriram que ouvir música é a segunda tática mais usada para regular as emoções, atrás apenas de “conversar com amigos”.

\section{PLAVRAS-CHAVES}

\begin{itemize}
 \item Emotion in music
 \item emotion regulation
 \item sentiment detection
 \item ranking
 \item music recommendation
 \item microblogging
 \item hashtags
 \item sentiment lexica
 \item DeepWalk
 \item Huffman tree construction
 \item fallback
 \item Gaussian mixture models 
\end{itemize}

\end{document}

