\documentclass{article}
\usepackage[utf8]{inputenc}
\title{Fichamento}
\author{GABRIEL PRÓSPERO REALIZADOR  SILVA}
\date{Abril 2021}

\usepackage{natbib}
\usepackage{graphicx}
\begin{document}
\maketitle

\section{DADOS DO ARTIGO}
\textbf{NOVEL HYBRID MODEL FOR MUSIC GENRE CLASSIFICARION BASED ON SUPPORT VECTOR MACHINE \\}
\author{Srishti Sharma \\}
\date{2018}

\section{RESENHA}
Neste artigo, um novo método de classificação de gêneros musicais é proposto com base no empilhamento de Support Vector Machine (SVM) com Relevance Vector Machine (RVM) e árvores de decisão.
A Rede Neural Convolucional tem sido utilizada para classificar os gêneros com base na representação visual de sinais [3].
Outra abordagem para esse problema era explorar a propriedade dos dados de serem séries temporais. O modelo LSTM-RNN foi usado para atender a esse fator de dados [4]. O uso de Redes Neurais Profundas (DNN) auxilia no treinamento de grandes quantidades de dados [5-8]. O desenvolvimento do DNN levou à sua aplicação ativa neste campo [9-10]. Outras técnicas utilizadas incluem k-Nearest Neighbor [11], que foi superado por SVM [12]. Neste artigo, um modelo híbrido de SVM, RVM e um classificador de conjuntos de árvores de decisão é usado para classificar gêneros musicais.
Os recursos acústicos dos sinais de áudio são amplamente classificados em duas categorias; Perceptivo, baseado na maneira como o áudio é ouvido por humanos como Rhythm, Pitch e Timbre e Physical, baseado em propriedades estatísticas e matemáticas de sinais como Zero Crossing Rate (ZCR), energia e frequência [13].
Para calcular os MFCCs de um sinal, ele é dividido em vários quadros curtos para manter o sinal constante. Em seguida, as estimativas do periodograma do espectro de potência são calculadas para todos os quadros para identificar as frequências presentes nos quadros. Em seguida, o banco de filtros com espaçamento de Mel é calculado e a energia em cada filtro é somada. A quantidade de energia em várias regiões de frequência é agora conhecida. Para ter as características mais próximas do que os humanos ouvem, o logaritmo dessas energias é calculado.
Os descritores de espectro estatístico (SSD) são vitais para computar momentos estatísticos em bandas críticas geradas usando o espectrograma da escala de Bark, bem como para delinear as variações de energia rítmica observadas nessas regiões críticas. Esses sete momentos estatísticos mencionados são calculados: média, mediana, variância, assimetria, curtose, valores mínimo e máximo para cada banda crítica agrupada no espectrograma de escala de Bark que compõe o conjunto de recursos SSD. A escala de Bark é convertida em escala de decibéis. Esses valores são posteriormente alterados para produzir valores Sone equivalentes antes que esses momentos sejam finalmente calculados no espectrograma resultante após as transformações. O conjunto final de recursos SSD de um arquivo de áudio é a mediana de todos os segmentos extraídos do sinal de áudio fornecido.
Os recursos de histograma de ritmo se mostram muito úteis em termos de coleta de características rítmicas de um arquivo de áudio. Em contraste com recursos como SSD, o processamento não é feito em todas as bandas críticas, em vez disso, um histograma é formado por 60 bins, somando as frequências de modulação para todas as 24 bandas críticas com frequências que variam de 0,168 Hz a 10 Hz. Semelhante ao SSD, o conjunto de recursos RH também é a mediana dos respectivos histogramas de todos os segmentos processados.
O espectro de qualquer sinal de áudio é indicativo de como as frequências são distribuídas dentro do sinal capturado. A média ponderada de todas essas frequências é o centróide espectral desse sinal particular. Diferentes gêneros e subgêneros musicais geralmente têm centróides espectrais característicos que podem ser um fator crucial no estabelecimento da linha de fronteira.
O roll off espectral é um recurso confiável para medir a forma do sinal de entrada. No espectro de potência, o roll of point espectral pode ser calculado determinando a frequência limite ou o número de bins em que 80 por cento da energia é distribuída abaixo desta frequência [14]. O ponto de roll off espectral pode ser encontrado pela seguinte equação, onde Mt [n] é a Transformada de Fourier para t-ésimo quadro e enésima categoria de frequência.

\section{PLAVRAS-CHAVES}
\begin{itemize}
    \item Convolutional Neural Network
    \item ROC curve
    \item RVM
    \item SVM
    \item Mel Frequency Cepstral Coefficients (MFCC)
\end{itemize}

\end{document}

