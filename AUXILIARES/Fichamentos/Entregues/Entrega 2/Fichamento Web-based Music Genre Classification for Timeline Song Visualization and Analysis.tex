\documentclass{article}
\usepackage[utf8]{inputenc}
\title{Fichamento}
\author{GABRIEL PRÓSPERO REALIZADOR  SILVA}
\date{Março 2021}

\usepackage{natbib}
\usepackage{graphicx}
\begin{document}
\maketitle

\section{DADOS DO ARTIGO}
\textbf{Web-based Music Genre Classification for Timeline Song Visualization and Analysis \\}
\author{JAIME RAMIREZ CASTILLO \\}
\date{DEZEMBRO DE 2020}

\section{RESENHA}
Os classificadores Naive Bayes (NB) pertencem à família dos modelos probabilísticos. Em particular, eles são categorizados dentro de Redes Bayesianas, que apresentam dois componentes principais: (1) a representação do modelo gráfico - DAG (Directed Acyclic Graph) - com um conjunto de nós e arestas, e (2) uma distribuição de probabilidade conjunta (JPD) . Os nós representam as variáveis no domínio do problema e as arestas referem-se a seus relacionamentos diretos.
No caso do tópico MIR, os RNNs têm sido usados ​​na análise musical [25], criação musical [39] ou reconhecimento de acordes [40].
Os autores usaram um conjunto de dados de 30 vetores de recursos extraídos de sinais de áudio para prever o gênero musical entre 10 opções diferentes. Experimentando com o modelo de mistura gaussiana (GMM) e classificadores k-vizinhos mais próximos (k-NN), os autores alcançaram valores de precisão de 61%.
O interesse e progresso no campo MGC é notável, com mais de 500 publicações relatadas [42]. No entanto, o campo ainda apresenta desafios em aberto, como a definição pouco clara do gênero musical. Atualmente, o conceito está sujeito a diferentes perspectivas e opiniões e vagamente definido [43]. Existem várias taxonomias publicamente disponíveis que classificam os gêneros musicais, mas carecem de acordo sobre as definições e descritores de gênero [43].
Recursos de áudio baseados em conteúdo são extraídos de sinais brutos. Esses recursos descrevem o áudio em termos de altura, timbre ou ritmo, entre outros descritores, e geralmente são extraídos usando a Short Time Fourier Transform (STFT) [52].
Uma característica tímbrica comumente usada são os coeficientes cepstrais de frequência de Mel (MFCCs). Os MFCCs capturam o espectro de sons e têm um bom desempenho em diferentes problemas de MIR [53], [54]. Outras características comuns são centróide espectral, rolloff espectral e cruzamentos de zero no domínio do tempo [5].
Não existe uma definição formal geralmente aceita para gênero musical. As categorizações disponíveis são definidas usando perspectivas arbitrárias (por exemplo, recursos musicais, categorização editorial, efeitos emocionais ou culturais), mudança ou sobreposição de definições de gênero e níveis em taxonomias hierárquicas são confusos [78]. Embora todas essas categorizações de gênero diferentes mostrem pouco consenso, eles não são conjuntos disjuntos completamente diferentes. Mapear gêneros musicais entre diferentes fontes é tão complexo que as próprias CNNs foram aplicadas para resolver essa tarefa [79].
\section{PLAVRAS-CHAVES}
\begin{itemize}
    \item Naive Bayes
    \item ID3 Algoritm
    \item Fully Connected Neural Networks (FCNNs)
    \item deep artificial neural networks (ANNs)
	\item RNNs
	\item MGC
	\item STFT
	\item Deep Belief Networks (DBN)
	\item Audioset
	\item Mel Frequency Cepstral Coefficients (MFCCs)
	\item embeddings
	\item VGGish
	\item sigmoid function.
\end{itemize}

\end{document}

