\documentclass{article}
\usepackage[utf8]{inputenc}
\title{Fichamento}
\author{GABRIEL PRÓSPERO REALIZADOR  SILVA}
\date{Março 2021}
\usepackage{natbib}
\usepackage{graphicx}
\begin{document}
\maketitle

\section{DADOS DO ARTIGO}
\textbf{COMPUTER-MUSIC INTERFACES: A SURVEY \\}
\author{PENYCOOK \\}
\date{1895}

\section{RESENHA}
Muitos dos requisitos de interface para um sistema de música são exclusivos da música e não têm relevância no domínio geral do uso do computador. Os requisitos do sistema para interfaces de música variam de acordo com a tarefa e com o natureza do ambiente musical. Em um ambiente de tempo real, existe a seguinte hierarquia de requisitos de tempo: (1) controle de estruturas musicais formais, como instrumentação e número de canais de saída; (2) execução de desempenho a partir dos dispositivos de entrada; (3) fundir o desempenho e dados de síntese em um fluxo contínuo e transmiti-los para o dispositivo de síntese; (4) cálculo do amostras de som; (5) saída de e para subsistemas de conversão de áudio. Integrar os tempos de resposta mínimos em todos os níveis do sistema, pode representar um problema de taxa de transferência Uma tendência para computadores pequenos e poderosos baseados em UNIX é fornecer os meios para produzir estações de trabalho de áudio especializadas com recursos anteriormente limitados a grandes computadores.

\section{PLAVRAS-CHAVES}
\begin{itemize}
    \item COMPOSITION AND SYNTHESIS LANGUAGES COMPUTER; 
    \item AIDED INSTRUCTION IN MUSIC SYSTEMS; 
    \item DESIGN PRINCIPLES; 
    \item GRAPHIC SCORE EDITING; 
    \item REAL-TIME PERFORMANCE SYSTEMS; 
    \item TECHNICAL CONSIDERATIONS; 
    \item MUSIC USER INTERFACE CONCEPTS; 
    \item DESIGN STRATEGIES; 
    \item SYSTEMS SURVEY;
\end{itemize}

\end{document}
