\documentclass{article}
\usepackage[utf8]{inputenc}
\title{Fichamento}
\author{GABRIEL PRÓSPERO REALIZADOR  SILVA}
\date{Março 2021}
\usepackage{natbib}
\usepackage{graphicx}
\begin{document}
\maketitle

\section{DADOS DO ARTIGO}
\textbf{EMOTION BASED MUSIC RECOMMENDATION SYSTEM \\}
\author{MIKHAIL RUMIANTCEV \\}
\date{2020}

\section{RESENHA}
Este artigo apresentará o design do sistema de recomendação musical personalizado, impulsionado pelos sentimentos, emoções e contextos de atividade do ouvinte. Com uma combinação de tecnologias de inteligência artificial e abordagens generalizadas de musicoterapia, um sistema de recomendação é direcionado para ajudar as pessoas na seleção de músicas para diferentes situações de vida. Os humanos percebem uma variedade de sentimentos em diferentes tipos de música e desde os tempos antigos, a mesma é considerada a influência da música na formação de um caráter pessoal e capacidade de tratar doenças [1]. O sistema precisa estar ciente dos eventos da experiência de escuta, sua duração, preferências, feedback e flutuações físicas e psicofísicas. Para obter uma personalização profunda, o sistema precisa coletar e processar grandes quantidades de eventos individualizados em tempo real. É necessário um sistema distribuído para suportar grandes fluxos de dados com um número crescente de usuários. Apache Kafka é a plataforma de fluxo de mensagens que pode ser usada como ferramenta para vincular aplicativos clientes e unidades de processamento de dados, possui alto rendimento e baixa latência no fluxo de dados em tempo real. 

\section{PLAVRAS-CHAVES}
\begin{itemize}
    \item LONG SHORT-TERM MEMORY (TSTM); 
    \item MARCOV DECISION PROCESS;
\end{itemize}

\end{document}
