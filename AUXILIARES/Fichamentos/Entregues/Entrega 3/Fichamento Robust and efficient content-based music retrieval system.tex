\documentclass{article}
\usepackage[utf8]{inputenc}
\title{Fichamento}
\author{GABRIEL PRÓSPERO REALIZADOR  SILVA}
\date{Abril 2021}

\usepackage{natbib}
\usepackage{graphicx}
\begin{document}
\maketitle

\section{DADOS DO ARTIGO}
\textbf{Robust and efficient content-based music retrieval system \\}
\author{Yuan-Shan Lee \\}
\date{2016}

\section{RESENHA}
Representação do conteúdo musical Os MFCCs foram propostos pela primeira vez por Davis e Mermelstein em 1980 [13]. Os MFCCs são representações não paramétricas dos sinais de áudio e são usados ​​para modelar o sistema de percepção auditiva humana [9]. Portanto, MFCCs são úteis para reconhecimento de áudio [14]. Este método fez contribuições importantes na recuperação musical até o momento. Tao et al. [8] desenvolveu um sistema QBS usando a matriz MFCCs. Para melhorar a eficiência do sistema, um esquema de agrupamento em dois estágios foi usado para reorganizar o banco de dados. Por outro lado, o recurso Chroma proposto por Shepard [10] tem sido aplicado em estudos de recuperação musical com grande eficácia. Xiong et al. [15] propôs um sistema de recuperação de música que usava o recurso Chroma e a tecnologia de detecção de notas. O conceito principal deste sistema é extrair uma impressão digital de música do recurso Chroma. Sumi et al. [16] propôs um sistema de recuperação baseado em símbolo que usa o recurso Chroma e recursos de pitch para construir consultas. Além disso, para tornar o sistema com alta precisão, campos aleatórios condicionais têm sido usados ​​para aprimorar recursos. Os recursos do Chroma podem funcionar bem quando consultas e dados de referência são reproduzidos em partituras musicais diferentes. Verificou-se que os recursos do Chroma podem identificar músicas em diferentes versões. Portanto, podemos usar os recursos do Chroma para identificar todos os tipos de músicas, até mesmo versões cover [17]. Esta pesquisa estende nosso trabalho anterior [6].
O conceito de entropia de informação foi introduzido pela primeira vez por Claude E. Shannon em 1948.

\section{PLAVRAS-CHAVES}
\begin{itemize}
    \item content-based music retrieval  (CBMR) 
    \item Query-by-singer (QBS)
    \item MFFCs
    \item QBS
    \item AFAA
\end{itemize}

\end{document}

