\documentclass{article}
\usepackage[utf8]{inputenc}
\title{Fichamento}
\author{GABRIEL PRÓSPERO REALIZADOR  SILVA}
\date{Março 2021}

\usepackage{natbib}
\usepackage{graphicx}
\begin{document}
\maketitle

\section{DADOS DO ARTIGO}
\textbf{Gaussian Processes for Music Audio Modelling and Content Analysis \\}
\author{Alvarado, Pablo A. and Stowell, Dan \\}
\date{2016}

\section{RESENHA}
AMT se refere à extração de uma descrição legível e interpretável por humanos de uma gravação de uma apresentação musical. Nos referimos ao AMT polifônico nos casos em que mais de um único tom musical é reproduzido em um determinado instante de tempo.
A tarefa geral de interesse é inferir automaticamente uma nota P.A.A musical graças a Colciencias pelo financiamento. , como a notação musical tradicional ocidental, listando os valores de pitch das notas, timestamps correspondentes e outras informações expressivas em um determinado sinal de áudio de uma performance [4]. 
Normalmente, os algoritmos para AMT são desenvolvidos de forma independente para realizar tarefas individuais, como detecção de F0 múltiplo, rastreamento de batida e reconhecimento de instrumento. O desafio permanece em combinar esses algoritmos, para realizar a estimativa conjunta de todos os parâmetros [3].
Os GPs têm sido usados ​​também para estimar o envelope espectral e a frequência fundamental de um sinal de voz [19].
Mostramos quais kernels eram mais apropriados para descrever propriedades de sinais musicais, especificamente: não estacionariedade, dinâmica e conteúdo harmônico espectral. A vantagem dessa abordagem é que, ao projetar um kernel adequado, podemos introduzir conhecimentos e crenças anteriores sobre as propriedades dos sinais musicais e usar todas as informações anteriores para melhorar a previsão.
\section{PLAVRAS-CHAVES}
\begin{itemize}
    \item AMT
\end{itemize}

\end{document}

