\documentclass{article}
\usepackage[utf8]{inputenc}
\title{Fichamento}
\author{GABRIEL PRÓSPERO REALIZADOR  SILVA}
\date{Abril 2021}

\usepackage{natbib}
\usepackage{graphicx}
\begin{document}
\maketitle

\section{DADOS DO ARTIGO}
\textbf{Multi Genre Music Classification and Conversion System \\}
\author{Siddavatam, Irfan and Dalvi, Ashwini and Gupta, Dipen and Farooqui, Zaid and Chouhan, Mihir \\}
\date{2020}

\section{RESENHA}
Os sistemas de classificação de gênero musical têm sido uma grande parte da sociedade musical por mais de uma década. Os sistemas de classificação de gênero são usados ​​para identificar o gênero do arquivo de música fornecido pelo usuário. Gênero é o domínio musical ao qual a canção pertence viz. Música Pop, Música Rock, Música Clássica e assim por diante. Trabalhos anteriores foram realizados neste campo usando várias abordagens, que vão do aprendizado de máquina a redes neurais.
O objetivo desta tarefa é extrair informações de áudio relevantes. Isso ajuda a reduzir a dimensionalidade dos dados. Isso simplifica o processo de aprendizagem. É importante acertar essa tarefa, pois ela influencia muito o modelo de aprendizagem. O pré-processamento de áudio deve ser capaz de extrair dados da música. O objetivo desta tarefa é extrair e organizar os dados de uma maneira que ajude o modelo a ter um bom desempenho.
Uma etapa importante no desenvolvimento de um modelo preciso e de bom desempenho é o treinamento do modelo. Existem vários parâmetros que influenciam muito a saída e a precisão do modelo. Os parâmetros como tamanho do lote, épocas, otimizador, perda, etc, têm efeito na saída do modelo.
Converta o arquivo de entrada em seu espectrograma usando STFT (Short Time Fourier Transform).
O resultado é verificado posteriormente pelo método de pesquisa. Quase todo mundo tem sua própria perspectiva diferente sobre a música, e o que torna um gênero musical diferente do outro.
\section{PLAVRAS-CHAVES}
\begin{itemize}
    \item STFT
    \item Conversão de genêro
\end{itemize}

\end{document}

