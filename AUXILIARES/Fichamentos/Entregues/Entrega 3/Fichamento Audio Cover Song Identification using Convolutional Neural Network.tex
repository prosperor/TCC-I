\documentclass{article}
\usepackage[utf8]{inputenc}
\title{Fichamento}
\author{GABRIEL PRÓSPERO REALIZADOR  SILVA}
\date{Abril 2021}

\usepackage{natbib}
\usepackage{graphicx}
\begin{document}
\maketitle

\section{DADOS DO ARTIGO}
\textbf{ Audio Cover Song Identification using Convolutional Neural Network  \\}
\author{Chang, Sungkyun and Lee, Juheon and Choe, Sang Keun and Lee, Kyogu \\}
\date{2017}

\section{RESENHA}
Até agora, houve algumas tentativas de explorar o aprendizado de máquina para a identificação de covers. Humphrey et al. [2013] usaram codificação esparsa com coeficiente de magnitude de Fourier bidimensional derivado de croma. Recentemente, Heo et al. [2017] tentaram aplicar o aprendizado métrico (Davis et al. [2007]) aos resultados do SimPLe.
Com base em Hu et al. [2003], primeiro convertemos os sinais de áudio de cada música em um recurso croma de 12 dimensões com uma janela não sobreposta de 1 s. Então, podemos definir uma matriz de similaridade cruzada S em relação a um par de dois recursos de croma {A, B}
Ao classificar a saída do softmax da CNN treinada, o sistema proposto foi capaz de prever um número fixo dos pares de canções cover mais prováveis. O desempenho do sistema proposto foi comparado com uma abordagem determinística e outra baseada em aprendizado de máquina.

\section{PLAVRAS-CHAVES}
\begin{itemize}
    \item CNN
    \item COVER
    \item CONVOLUCIONAL
\end{itemize}

\end{document}

